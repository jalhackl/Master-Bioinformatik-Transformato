\chapter{Conclusion}

The results assessed so far are highly ambiguous. In most cases {\color{red} (NAJA, es sind eigentlich nur 2 cases bis jetzt, wenn man den ersten, wahrscheinlich fehlerhaften run wegzaehlt und davon ausgeht, dass der zweite Versuch funktioniert fuer Toluen und Naphthol funktioniert hat, was jedenfalls noch ueberprueft werden sollte - auch hier gibt es einige Zweifel!!!)}, no profound differences have been found. In any case, additional comparisons of mutation routes for other molecules and closer examination of the obtained results are necessary.

Perhaps the most promising approach would be a simulation for the molecule pairs used in \cite{Loeffler.2018, Wieder.2022}. (This would also allow a direct comparison with the results stated therein and a validity check of the new results.) Routes for the most interesting (i.e., molecules with non-trivial mutation orders involving rings etc.) molecule pairs (toluene, 2-methylfuran and 2-methylindole to methane and 2-cyclopentylindole to 7-cyclopentylindole) can be found in the appendix 7.3. 
On the other hand, a more systematic evaluation of the emergence of differences between the mutation algorithms could use molecules with exactly defined, small modifications (for example, one added/removed atom or ring structure) to investigate if the size of the dummy regions or the common core and the combination of some elements, e.g. rings, have an effect.


A further path for detecting differences between the outcome of the mutation algorithms is to compare runs of different sampling length. The simulation length used so far gives good results for any mutation order, but for shorter simulations it could be that differences increase. It could be that the increase in standard deviation which is expected for short sampling lengths is different for different mutation orders. There are several possible outcomes: Either the standard deviation increases slower for one of the algorithms with decreased simulation length, which suggests its superiority, or the resulting free energy differences diverge indicating the importance of the mutation route at least. Finally, it could be that the changes in the free energy results occur at the same simulation length for both algorithms implying that there are no significant differences between the mutation orders.
For toluene, simulations with the same set-up, but of shorter length have already been carried out; however, the results do only slighter differ. Basically, the shorter run for toluene$ 	\rightarrow $methanol showed identical results (but, as expected, a slightly higher standard deviation) for both algorithms. It would be necessary to decrease the sampling length even more until the results for different mutation orders start do differ significantly.

Especially for more complex transformations an evaluation of the relation between sampling length and standard deviation for different mutation routes could be insightful. (Choosing an algorithm which allows for shorter sampling size by minimizing variance would be advantageous as well.)

Given the results obtained so far, it seems that there is no general superiority of a more ‘symmetric’, breadth-first-based algorithm. (For toluene and 2-naphthol, the results for the new algorithms are even worse measured as increase in standard deviation. For both molecules, it is higher for the new algorithm, in the case of 2-naphthol the increase is above 300\%.) More data would be required to get more conclusive results (e.g. to check if the 'best' way of processing a ring system depends on its substituents or if the size of the atom is influential etc.). Nonetheless, the results partly indicate that the mutation order could play a more complex role. Although the new algorithm leading to a more 'logical' and systematic processing doesn't give better results, the mutation order seems to influence the free energy estimation, especially the standard deviation for a given simulation length.

The tf-routes package presented in this work also allows to process heavy non-carbon atoms in an individual manner, i.e. atoms of different types can be represented as nodes with different weight which impacts the mutation route. It would be interesting if there are more pronounced differences for molecules with such atoms.

However, in any case more results for molecules with different structure (e.g. different size and number of rings and chains, possible different atoms systems) and different sampling length would be necessary. The molecule pairs assessed so far (the mutation route between methanol and toluene or naphthol) are rather small and do not cover the space of possible structures at all.
Elucidating the apparently more complex and specific role between mutation order and resulting free energy differences could allow further improvement of the mutation route creation (given that the overall differences suggest that further optimization is expedient).\\
Perhaps it would be useful to apply different methods for optimizing the mutation route, e.g. learning the route which minimizes the variance per step via machine learning etc. (there are several possibilities, e.g. provide the molecule - as graph representation etc. - and changes in free energy for one atom turned off as training set etc.).

