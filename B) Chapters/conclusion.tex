\chapter{Conclusion}


Simulation results for the molecule pairs used in \cite{Loeffler.2018, Wieder.2022} demonstrate the superiority of the new mutation algorithms. In each case, the standard deviation of the new algorithms is smaller and for 2-/7-CPI the free energy difference is closer to the result of absolute free energy calculations. 
In this report, only tests on rather small molecules and simple transformations have been carried out. Additional comparisons of mutation routes for more molecules and closer examination of the obtained results are necessary.

Especially for more complex transformations, an evaluation of the relation between sampling length and standard deviation for different mutation routes could be insightful. 
In such cases, differences between protocols might be more pronounced, and choosing an algorithm which allows for shorter sampling size by minimizing variance would lower the computational cost.

To investigate such correlations systematically, one could apply the mutation algorithms to a series of molecules with exactly defined, small modifications (for example, one added/removed atom or ring structure). This would allow one to probe if the size of the dummy regions or the common core and the combination of some elements, e.g., rings, have an effect. However, it should be stressed that in most cases the effects probably will be moderate and smaller than the estimated standard error of the simulations.


The tf-routes package presented in this work also allows processing heavy non-carbon atoms with specific priorities; i.e., atoms of different types can be represented as nodes with different weight, which impacts the mutation route. It would be interesting if there are more pronounced differences for molecules with such atoms.

However, in any case, more results for molecules with different structure (e.g., different size and number of rings and chains, possible different atoms systems) and different sampling length would be necessary. The molecule pairs assessed so far are rather small and do not cover the space of possible structures at all.
Elucidating the apparently more complex and specific role between mutation order and resulting free energy differences could allow further improvement of the mutation route creation (given that the overall differences suggest that further optimization is expedient).

The greatest advantage of the new algorithms in comparison to the old versions is certainly that they always should yield a correct common core and a 'reasonable' route for every molecule pair (if a correct common core exists). In particular, the lack of special treatment of common core hydrogen atoms produced faulty common cores in previous versions of {\trafo}. The new processing always generates a valid common core with maximum atom size. Furthermore, the old mutation route algorithm often led, especially in the case of bigger molecules, to atom removals in regions near the common core before chains etc. were systematically processed. Now both parts of the workflow, common core generation and mutation route determination, are adjusted and should always yield reasonable and efficient mutation routes.
