\chapter{{\trafo} }

{\trafo} uses a common core scaffold which contains the subset of atoms which are present in both molecules (i.e., a one-to-one correspondence between
atoms of both molecules, which in the test cases shown below is always
based on atom identity). Both initial states of the alchemical transformations initialized by {\trafo} do not contain any dummy atoms, but consist solely of the physical atoms of the respective molecules. However, dummy atoms are generated via two separate alchemical paths leading to the common core. Starting from the initial states, physical atoms are successively turned into dummy atoms until the common core structure is reached. 

In each transformation step, one physical atom (or, if phase space
overlap is sufficient, a batch of adjacent atoms) is changed into
a dummy atom (until the common core is attained).

The common core architecture circumvents some potential problems
associated with the single and double topology approaches for dummy
atoms. The physical end states of the molecules are mutated until the
common core structure is attained. This implies that both starting states
do not contain any dummy atoms (these states are identical to the
physical molecules of interest). Therefore, it is ambiguous if the
alchemical transformations implemented in {\trafo} rather belong
to the single or the dual topology paradigm: As in the latter, different
dummy atoms are generated for each molecule along the path to the
common core, but --- in contrast to the usual dual topology approach
--- the final states are free of any dummy atoms. 

The 'removal' of atoms, i.e., the mutation into dummy atoms, is performed
using the serial atom insertion approach described above; hence, {\trafo}
does not rely on the use of soft core potentials. The computation
of the resulting energy differences is based on Bennett's acceptance
ratio method described above.

Therefore, by setting up the mutation path between two molecules across
the connecting common core and using serial atom
insertion, the {\trafo} workflow is independent of the underlying
molecular dynamics package and of the availability of explicit free energy
calculation code. In principle, {\trafo} can work on top of every
molecular dynamics simulation package.

Input can be created via CHARMM-GUI\cite{Jo.2008}. The solution builder of the website generates appropriate files to run MD simulations for the systems for which free energy differences shall be determined (e.g., for relative solvation free energy differences, input files for both molecules in a water box and in vacuum).\cite{Braunsfeld., Karwounopoulos.2022}.

\section{Common core approach}

The main condition for the common core of two molecules is the existence
of a one-to-one correspondence of atoms, i.e. the existence of a graph
isomorphism. The {\trafo} workflow imposes some further conditions
on the properties of the common core:

The junction between the common core and dummy region has to be unique;
only one dummy region is allowed to be connected via one bond to the
common core. In particular, the maximum common substructure must not
encompass partial rings (which would imply that dummy regions are
connected via multiple bonds). (In general, such transformations can pose intricate problems because ring breakage can lead to fast changes in free energy
and cause significant estimation error \cite{Liu.2015}.)
To meet these requirements, adjusting of the parameters and algorithms for finding the common core is important. For facilitating the construction of appropriate common cores, it was also necessary to improve the processing of the hydrogen atoms. As hydrogens are turned off beforehand in one step and not via serial atom insertion like the heavy atoms, it would be  detrimental to consider them in the common core generation (see section 'Processing of hydrogen atoms' in the next chapter). In previous versions of {\trafo}, hydrogens were included in the common core generation steps which led to small or in many cases even totally infeasible common cores. Before the common core construction is carried out, hydrogens have to be
removed from the molecule representation. This is an important step
because the presence of hydrogen atoms can lead to a different common
core (which is created, using default settings, by maximizing the
number of corresponding atoms) and subsequently a suboptimal mutation
route. 




\section{{\trafo} workflow}


During the mutation process, the contributions of the non-CC atoms are turned off gradually; five stages can be discerned \cite{Karwounopoulos.2022}:  In the first step, the electrostatic interactions of dummy atoms are gradually turned off. Next, Lennard-Jones interactions of all hydrogen atoms outside the common core are removed. This can be carried out in a single step.
During the third stage, the LJ-interactions of the non-CC non-hydrogen atoms are processed. One atom per step is turned off following the serial atomic insertion approach. (Possibly, a small group of atoms could be turned off in one step, but it is not recommended to process more than two atoms at once to ensure sufficient phase space overlap between the states.)
The atom which connects the common core and the dummy region is  called the junction atom and indicated as 'X' \cite{Karwounopoulos.2022} (In the plots in the next chapters, for simplicity, the indication of the junction atom is sometimes omitted. However, it should be stressed that, e.g., a transformation to a methane common core yields, in fact, a $\mathrm{CH_3X}$ common core.). This dummy atom which is directly connected to the common core needs special treatment. In contrast to the other atoms of a dummy region, its LJ interactions are not turned off completely, though its partial charge has to be zero.\cite{Karwounopoulos.2022} This junction atom ensures that the double free energy differences are not influenced by contributions of dummy atoms.\cite{Fleck.2021}
In the last step, it has to be ensured that both common cores are identical, e.g., differences in charge distribution have to be adjusted and the bond between junction atom and the common core atom has to be identical.

