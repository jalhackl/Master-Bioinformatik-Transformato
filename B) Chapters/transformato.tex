\chapter{Transformato }

Transformato uses a common core scaffold which contains the atoms
present in both molecules (i.e., a one-to-one-correspondence between
atoms of both molecules which in the test cases shown below is always
based on atom identity). Both initial states of the alchemical transformations
initialized by Transformato do not contain any dummy atoms, but consist
solely of the physical atoms of the respective molecules. However,
dummy atoms are generated via two separate alchemical paths leading
to the common core. Starting from the initial states, physical atoms
are successively turned into dummy atoms until the common core structure
is attained. 

In each transformation step one physical atom (or, if phase space
overlap is sufficient, a batch of adjacent atoms) is changed into
a dummy atom (until the common core is attained).

The common core architecture circumvents some of the potential problems
associated with the single and double topology approaches for dummy
atoms. The physical endstates of the molecules are mutated until the
common core structure is attained. This implies that both final states
do not contain any dummy atoms (these states are identical to the
physical molecules of interest). Therefore, it is ambiguous if the
alchemical transformations implemented in Transformato rather belong
to the single or the dual topology paradigm: As in the latter, different
dummy atoms are generated for each molecules along the path to the
common core, but - in contrast to the common dual topology approach
- the final states are free of any dummy atoms. 

The 'removal' of atoms, i.e. the mutation into dummy atoms, is performed
according to the serial atom insertion described above, hence Transformato
does not rely on the usage of softcore potentials. The computation
of the resulting energy differences is based on Bennett's acceptance
ratio described above.

Therefore, setting up the mutation path between two molecules across
the connecting common core and going along by means of serial atom
insertion, the Transformato workflow is independent of the underlying
molecular dynamics package and the availability of explicit free energy
calculation code. In principle, Transformato can work on top of every
molecular dynamics simulation package.

Input can be created via CHARMM-GUI\cite{Jo.2008}\cite{Braunsfeld.}.

\section{Common core approach}

The main condition for the common core of two molecules is the existence
of a one to one correspondence of atoms, i.e. the existence of a graph
isomorphism. The Transformato workflow imposes some further conditions
on the properties of the common core:

So the junction between common core and dummy region has to be unique;
one dummy region is only allowed to be connected via one bond to the
common core. In particular, the maximum common substructure must not
encompass partial rings (which would imply that dummy regions are
connected via multiple bonds).

During the mutation process, the atoms are turned off gradually: In
a first step, all hydrogen atoms outside the common core are excluded.
The electrostatic interactions of dummy atoms are turned off. In a
next step, the van-der-Waals-interactions are processed (in one step
per atom in line with serial atomic insertion). 

Before the common core construction is carried out, hydrogens are
removed from the molecule representation. This is an important step
because the presence of hydrogen atoms can lead to a different common
core (which is created, using default settings, by maximizing the
number of corresponding atoms) and subsequently a flawed mutation
route. For the workflow of Transformato, inclusion of hydrogens in
the mutation algorithms would be detrimental because these atoms are
already turned off beforehand.