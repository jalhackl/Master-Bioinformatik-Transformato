\chapter{Introduction}

The aim of this Master Thesis is to facilitate the preparation of
alchemical free energy calculations. Such calculations
estimate free energy differences by using nonphysical intermediates, i.e., structures
which are not found in nature as existing chemical species. In addition
to the computation of absolute solvation and binding free energy differences,
the method can be used to compute relative free energy differences, e.g.,
the free energy difference of binding between two ligands. A problem occurring
in the latter approach is the need for so-called \textquoteleft dummy
atoms\textquoteright{}. Usually, the number of atoms between the two
end states, i.e., the two molecules of interest, is not the same.
However, this is a necessary condition for the molecular dynamics
simulations on which the computation of the free energy differences
is based. To preserve the number of atoms, these dummy atoms act as
placeholders\cite{Fleck.2021, Karwounopoulos.2022}.

This Master Thesis is concerned with specific methods of handling these nonphysical
atoms. A central part is the implementation of new features
for \trafo, a package which helps to set up relative alchemical
free energy calculations using an innovative common core approach\cite{key-2, Wieder.2022}.
In particular, helper functions are developed which optimize the employment of
the aforementioned dummy atoms. 
The implemented functions are collected in the  Python package tf-routes available on GitHub (https://github.com/jalhackl/tf\_routes/tree/master/tf\_routes).

In the following chapter, the basic principles of alchemical free energy
calculations are explained. The third chapter presents the workflow
of \trafo. Subsequently, the need for improving and extending some algorithms of the software
package is described in more detail. Examples for alchemical mutations
proposed by the new algorithms and corresponding common core constructions
for \trafo are given. Finally, the effect of different common core generation and mutation
algorithms on the results of free energy calculations is discussed.
