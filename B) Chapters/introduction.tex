\chapter{Introduction}

The aim of this Master Thesis is to facilitate the preparation of
alchemical free energy calculations. Such free energy calculations
estimate free energies by using unphysical intermediates, i.e. structures
which are not found in nature as existing chemical species. In addition
to the computation of absolute solvation and binding free energy differences,
the method can be used to compute relative free differences, e.g.,
the energy difference of binding between two ligands. A problem occurring
in the latter approach is the need for so-called \textquoteleft dummy
atoms\textquoteright . Usually, the number of atoms between the two
end states, i.e., the two molecules of interest, is not the same.
However, this is a necessary condition for the molecular dynamics
simulations on which the computation of the free energy differences
is based. To preserve the number of atoms these dummy atoms act as
placeholders\cite{Fleck.2021}. 

This Master Thesis works on specific methods of employing these unphysical
atoms. A central part will consist in the implementation of new features
for Transformato, a package which helps to set up relative alchemical
free energy calculations using an innovative common core approach\cite{key-2}.
In particular, additional functions will optimize the employment of
the aforementioned dummy atoms.

In the next chapter, the basic principles of alchemical free energy
calculations are explained. The third section presents the workflow
of Transformato. Subsequently, the tasks for improving the software
package are described in more detail. Examples for alchemical mutations
proposed by the new algorithms and corresponding common core constructions
for Transformato are given. Finally, the effect of different mutation
algorithms on the results of free energy calculations are discussed.